\documentclass[landscape, twocolumn, letterpaper,10pt, notitlepage]{article}

\usepackage[spanish]{babel}
\usepackage[utf8]{inputenc}

\usepackage{pdflscape}
\usepackage{graphicx}
\usepackage{tabularx}
\usepackage{slashbox}
\usepackage[intlimits]{amsmath}
\usepackage{amssymb}
\usepackage[left=2cm, right=2cm, top=4cm, bottom=3.5cm]{geometry}

\DeclareGraphicsExtensions{.jpg,.pdf,.mps,.png,.eps}

\newcommand{\ms}{\texttt}

\newcommand{\gra}[2]{
	\includegraphics[height=#2cm]{#1}
}

\newcommand{\grac}[2]{
	\begin{center}
	\includegraphics[height=#2cm]{#1}
	\end{center}
}

\begin{document}
\subsection{Ejemplo 1: 1 cambio elemental}

\begin{tabular}{||c|c|c|c||}
\hline
Media & Mediana & Desviación estándar & Rango intercuartil\\ \hline
0.7202& 0.7083& 0.2586& 0.4583\\\hline
\end{tabular}

\gra{E1.pdf}{10}

\subsection{Ejemplo 2: 3 cambios elementales}

\begin{center}
\begin{tabular}{||c|c|c|c||}
\hline
Media & Mediana & Desviación estándar & Rango intercuartil\\ \hline
0.5041& 0.5000& 0.1435& 0.1987\\\hline
\end{tabular}
\end{center}

\grac{E3.pdf}{12}

\subsection{Ejemplo 3: 4 cambios elementales}

\begin{center}
\begin{tabular}{||c|c|c|c||}
\hline
Media & Mediana & Desviación estándar & Rango intercuartil\\ \hline
0.5772& 0.5455& 0.1464& 0.2051\\\hline
\end{tabular}
\end{center}

\grac{E4.pdf}{12}

\subsection{Ejemplo 4: 6 cambios elementales}

\begin{center}
\begin{tabular}{||c|c|c|c||}
\hline
Media & Mediana & Desviación estándar & Rango intercuartil\\ \hline
0.4671& 0.4444& 0.1211& 0.1610\\\hline
\end{tabular}
\end{center}

\grac{E6.pdf}{12}

\end{document}
